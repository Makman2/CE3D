\documentclass{article}
\usepackage{amsmath}
\usepackage{amsfonts}
\usepackage[ngerman]{babel}

\newcommand{\rz}[1]{\ensuremath{\mathord{\mathrm{#1}}}}
\newcommand{\lrangle}[1]{\left\langle #1 \right\rangle}

\begin{document}

\title{Orthogonale Projektion}
\author{MAK}
\date{25.7.2014}


\maketitle






\section{Einleitung}

Eine orthogonale Projektion ist eine mathematische Funktion, die einen Vektor
\(\vec{v}\in\mathbb{R}\) auf eine (Hyper-)Ebene \(X\) projiziert, also den Vektor aus dem Raum
\(X\) abbildet, der die k"urzeste Distanz zum abzubildenden Vektor hat.

Dieser Artikel besch"aftigt sich im speziellen mit der dazugeh"origen Projektionsmatrix (f"ur die
Eingliederung in CE3D).






\section{Die orthogonale Projektion}

Eine orthogonale Projektion hat folgende Eigenschaften:

	\begin{itemize}
		\item Der projizierte Vektor \(p(\vec{x})\) ist eine Linearkombination aus den Basisvektoren
		      \(\vec{v}_1,\vec{v}_2,...,\vec{v}_m\) der Projektionsebene \(X\).
		\item Der projizierte Vektor abz"uglich des urspr"unglich zu projizierenden Vektors muss
		      senkrecht sein zur Projektionsebene (\(\lrangle{ p(\vec{x})-\vec{x},\vec{v} } =0\)).
	\end{itemize}

Es folgt eine Liste von Definitionen im Rahmen dieses Artikels:

	\begin{itemize}
		\item Der "ubergeordnete Vektorraum, in dem projiziert wird: \(V\). Die Dimension
		      des Raums ist definiert als \(dim(V)=n\).
		\item Die Projektionsebene: \(X=span(\vec{v}_1,\vec{v}_2,...,\vec{v}_m)\). Damit gilt f"ur
		      die Dimension der Ebene: \(dim(X)=m\).
		\item Die Projektionsfunktion: \(p(\vec{x})\).
		      \(p = \left\{\begin{array}{l}\mathbb{R}^n\to \mathbb{R}^n\\\vec{x}\mapsto
		      p(\vec{x})\end{array} \right.\)
	\end{itemize}






\section{Die Projektionsmatrix}
Um die Projektionsmatrix herzuleiten, ziehen wir die obigen Bedingungen heran und geben ihnen eine
verallgemeinerte mathematische Form:

	\begin{equation}
		\begin{array}{l l}
			\rz{I}  &
			p(\vec{x}) = \sum_{i=1}^{m}{\mu _i\vec{v_i}}, \quad \mu _i \in \mathbb{R} \\
			\rz{II} &
			\lrangle{ p(\vec{x}) - \vec{x}, \vec{v_i} } = 0, \quad \forall i \in \{ w \in
			  \mathbb{N} | 1 \leq w \leq m \}
		\end{array}
	\end{equation}

Die erste Bedingung l"asst sich in die zweite einsetzen und in ein Gleichungssystem umschreiben:

	\begin{equation}
		\lrangle{ \sum_{j=1}^{m}{\mu _j\vec{v_j}} - \vec{x}, \vec{v_i} } = 0, \quad \forall i
		  \in \{ w \in \mathbb{N} | 1 \leq w \leq m \}
	\end{equation}

	\begin{equation}
		\iff
		\begin{array}{c c c}
			\lrangle{ \mu _1 \vec{v}_1 + \mu _2 \vec{v}_2 + \cdots + \mu _m \vec{v}_m - \vec{x},
			  \vec{v}_1 } &
			= &
			0 \\
			\lrangle{ \mu _1 \vec{v}_1 + \mu _2 \vec{v}_2 + \cdots + \mu _m \vec{v}_m - \vec{x},
			  \vec{v}_2 } &
			= &
			0 \\
			\vdots &
			= &
			\vdots \\
			\lrangle{ \mu _1 \vec{v}_1 + \mu _2 \vec{v}_2 + \cdots + \mu _m \vec{v}_m - \vec{x},
			  \vec{v}_m } &
			= &
			0
		\end{array}
	\end{equation}

Alle konstanten Faktoren auf die rechte Seite bringen:

	\begin{equation}
		\iff
		\begin{array}{c c c}
			\lrangle{ \mu _1 \vec{v}_1 + \mu _2 \vec{v}_2 + \cdots + \mu _m \vec{v}_m,
			  \vec{v}_1 } & = & \lrangle{ \vec{x}, \vec{v}_1 } \\
			\lrangle{ \mu _1 \vec{v}_1 + \mu _2 \vec{v}_2 + \cdots + \mu _m \vec{v}_m,
			  \vec{v}_2 } & = & \lrangle{ \vec{x}, \vec{v}_2 } \\
			\vdots & = & \vdots \\
			\lrangle{ \mu _1 \vec{v}_1 + \mu _2 \vec{v}_2 + \cdots + \mu _m \vec{v}_m,
			  \vec{v}_m } & = & \lrangle{ \vec{x}, \vec{v}_m }
		\end{array}
	\end{equation}

Ausnutzen der Distributivit"at und Homogenit"at ergibt:

	\begin{equation}
		\iff
		\begin{array}{c c c}
			\mu _1 \lrangle{ \vec{v}_1, \vec{v}_1 } + \mu _2 \lrangle{ \vec{v}_2,
			  \vec{v}_1 } + \cdots + \mu _3 \lrangle{ \vec{v_3}, \vec{v}_1 } &
			= &
			\lrangle{ \vec{x}, \vec{v}_1 } \\
			\mu _1 \lrangle{ \vec{v}_1, \vec{v}_1 } + \mu _2 \lrangle{ \vec{v}_2,
			  \vec{v}_1 } + \cdots + \mu _3 \lrangle{ \vec{v_3}, \vec{v}_1 } &
			= &
			\lrangle{ \vec{x}, \vec{v}_2 } \\
			\vdots &
			= &
			\vdots \\
			\mu _1 \lrangle{ \vec{v}_1, \vec{v}_1 } + \mu _2 \lrangle{ \vec{v}_2,
			  \vec{v}_1 } + \cdots + \mu _3 \lrangle{ \vec{v_3}, \vec{v}_1 } &
			= &
			\lrangle{ \vec{x}, \vec{v}_m }
		\end{array}
	\end{equation}

Dieses Gleichungssystem kann durch eine Matrix wiedergegeben werden:

	\begin{equation}
		\iff
		\begin{pmatrix}
			\lrangle{ \vec{v}_1, \vec{v}_1 } &
			\lrangle{ \vec{v}_2, \vec{v}_1 } &
			\cdots &
			\lrangle{ \vec{v}_m, \vec{v}_1 } \\
			\lrangle{ \vec{v}_1, \vec{v}_2 } &
			\lrangle{ \vec{v}_2, \vec{v}_2 } &
			\cdots &
			\lrangle{ \vec{v}_m, \vec{v}_2 } \\
			\vdots & \vdots & \ddots & \vdots \\
			\lrangle{ \vec{v}_1, \vec{v}_m } &
			\lrangle{ \vec{v}_2, \vec{v}_m } &
			\cdots &
			\lrangle{ \vec{v}_m, \vec{v}_m }
		\end{pmatrix}
		\begin{pmatrix}
			\mu _1 \\
			\mu _2 \\
			\vdots \\
			\mu _m \\
		\end{pmatrix}
		=
		\begin{pmatrix}
			\lrangle{ \vec{v}_1, \vec{x} } \\
			\lrangle{ \vec{v}_2, \vec{x} } \\
			\vdots \\
			\lrangle{ \vec{v}_m, \vec{x} } \\
		\end{pmatrix}
	\end{equation}

Wir erhalten also die Gram'sche Matrix.
Nun m"ussen wir diese Gleichung so umformen, sodass wir eine Abbildung der Form

	\begin{equation}
		P\vec{x}=p(\vec{x})
	\end{equation}

bekommen mit \(P\) als zugeh"orige Projektionsmatrix.
Desweiteren definieren wir zur Vereinfachung die Gramsche Matrix und den Koeffizientenvektor:

	\begin{equation}
		G :=
		\begin{pmatrix}
			\lrangle{ \vec{v}_1, \vec{v}_1 } &
			\lrangle{ \vec{v}_2, \vec{v}_1 } &
			\cdots &
			\lrangle{ \vec{v}_m, \vec{v}_1 } \\
			\lrangle{ \vec{v}_1, \vec{v}_2 } &
			\lrangle{ \vec{v}_2, \vec{v}_2 } &
			\cdots &
			\lrangle{ \vec{v}_m, \vec{v}_2 } \\
			\vdots & \vdots & \ddots & \vdots \\
			\lrangle{ \vec{v}_1, \vec{v}_m } &
			\lrangle{ \vec{v}_2, \vec{v}_m } &
			\cdots &
			\lrangle{ \vec{v}_m, \vec{v}_m }
		\end{pmatrix}
	\end{equation}

	\begin{equation}
		\vec{\mu} :=
		\begin{pmatrix}
			\mu _1 \\
			\mu _2 \\
			\vdots \\
			\mu _m \\
		\end{pmatrix}
	\end{equation}

Unser Gleichungssystem wird nach \( \vec{\mu} \) umgeformt und in unsere erste Bedingung f"ur die
orthogonale Projektion eingesetzt (jeder projizierte Vektor muss eine Linearkombination der Basis
der Projektionsebene \(X\) sein):

	\begin{equation}
		G \vec{\mu} =
		\begin{pmatrix}
			\lrangle{ \vec{v}_1, \vec{x} } \\
			\lrangle{ \vec{v}_2, \vec{x} } \\
			\vdots \\
			\lrangle{ \vec{v}_m, \vec{x} } \\
		\end{pmatrix}
		\iff G^{-1}
		\begin{pmatrix}
			\lrangle{ \vec{v}_1, \vec{x} } \\
			\lrangle{ \vec{v}_2, \vec{x} } \\
			\vdots \\
			\lrangle{ \vec{v}_m, \vec{x} } \\
		\end{pmatrix} = \vec{\mu}
	\end{equation}

Bevor wir einsetzen wird die erste Bedingung auch in eine Matrixform "uberf"uhrt:

	\begin{equation}
		\sum_{i=1}^{m}{\mu _i \vec{v}_i} = p(\vec{x}) = 
		\lrangle{\vec{\mu},
		\begin{pmatrix}
			\vec{v}_1 \\
			\vec{v}_2 \\
			\vdots \\
			\vec{v}_m
		\end{pmatrix}
		} =
		\begin{pmatrix}
			\vec{v}_1 &&
			\vec{v}_2 &&
			\cdots &&
			\vec{v}_m
		\end{pmatrix}
		\vec{\mu}
	\end{equation}

Erneut definieren wir zur Vereinfachung:

	\begin{equation}
		A :=
		\begin{pmatrix}
			\vec{v}_1 &&
			\vec{v}_2 &&
			\cdots &&
			\vec{v}_m
		\end{pmatrix}
	\end{equation}

\(A\) ist also die Matrix, die die Basis unserer Projektionsebene \(X\) enth"alt.

Jetzt muss nur noch mathematisch umgeformt werden um die Projektionsmatrix \(P\) zu erhalten:

	\begin{equation}
		p(\vec{x}) = A\vec{\mu} = AG^{-1}
		\begin{pmatrix}
			\lrangle{ \vec{v}_1, \vec{x} } \\
			\lrangle{ \vec{v}_2, \vec{x} } \\
			\vdots \\
			\lrangle{ \vec{v}_m, \vec{x} } \\
			\end{pmatrix}
		= AG^{-1}
		\begin{pmatrix}
			\vec{v}_{1}^{T} \vec{x} \\
			\vec{v}_{2}^{T} \vec{x} \\
			\vdots \\
			\vec{v}_{m}^{T} \vec{x} \\
		\end{pmatrix}
		= AG^{-1}A^T x
	\end{equation}

Die Gram'sche Matrix l"asst sich darstellen durch

	\begin{equation}
		G = A^T A
	\end{equation}.

Damit haben wir unsere endg"ultige Projektionsmatrix zusammen:

	\begin{equation}
		p(\vec{x}) = A(A^T A)^{-1} A^T \vec{x}
	\end{equation}

	\begin{equation}
		P = A(A^T A)^{-1} A^T
	\end{equation}

mit unserer Basis-Matrix

	\begin{equation}
		A = \begin{pmatrix} \vec{v}_1 && \vec{v}_2 && \cdots && \vec{v}_m \end{pmatrix}
	\end{equation}.






\section{Projektion auf die Koeffizienten \(\vec{\mu}\) im Projektionsraum}

F"ur 3D-Anwendungen ist die Projektion auf die Koeffizienten mindestens genauso interessant wie
wichtig, z.B. um die Funktion einer Kamera zu implementieren, die einem die x- und y-Koordinaten
der Bildobjekte auf dem Sensor bzw. auf dem Display liefert.

Die Projektionsmatrix f"ur die Koeffizienten l"asst sich schnell herleiten aus der ersten
Projektionsbedingung (nach der "Ubersetzung in die Matrixform):

	\begin{equation}
		p(\vec{x}) = A \vec{\mu} = A(A^T A)^{-1} A^T \vec{x}
	\end{equation}

	\begin{equation}
		\iff \vec{\mu} = (A^T A)^{-1} A^T \vec{x}
	\end{equation}

Also ist die Projektionsmatrix f"ur die Koeffizienten:

	\begin{equation}
		P_{\mu} = (A^T A)^{-1} A^T
	\end{equation}






\section{Abstand von \(p(\vec{x})\) zu \(\vec{x}\)}

Der Abstand des projizierten Vektors zum urspr"unglichen Vektor ist von zentraler Bedeutung in der
Computergrafik. Um verschiedene dreidimensionale Objekte auf dem Bildschirm richtig arrangieren zu
k"onnen, m"ussen deren Abst"ande zur Projektionsebene bekannt sein. Ansonsten w"are bei
"Uberschneidungen von Objekten die 3D-Szene nicht richtig wiedergegeben.
Deshalb f"ugt man in die Matrix eine zus"atzliche Zeile ein, die die s.g. Tiefe berechnet.
Allerdings gibt es dabei ein Problem: Der euklidische Abstand der Vektoren ist eine nicht-lineare
Funktion! Damit ist sie auch nicht in einer Matrix darstellbar.

	\begin{equation}
		||\vec{x} - p(\vec{x})||_2 = \sqrt{\sum_{i=1}^{n}{(x_i - p(\vec{x})_i)^2}}, \quad \vec{x} =
		\begin{pmatrix}
			x_1 \\
			x_2 \\
			\vdots \\
			x_n
		\end{pmatrix}
	\end{equation}

Wir ben"otigen eine Funktion, die den Abstand wiedergibt und eine lineare Funktion ist. Deswegen
verwenden wir die 1-Norm (und ignorieren den komponentenweisen Betrag):

	\begin{equation}
		||\vec{x} - p(\vec{x})||_1 = \sum_{i=1}^{n}{x_i - p(\vec{x})_i} = \begin{pmatrix} 1 && \cdots && 1 \end{pmatrix} (\vec{x} - p(\vec{x}))
	\end{equation}

Es existiert eine Matrixform f"ur diese Abbildung, damit ist sie f"ur unsere Zwecke geeignet.
Anmerkung: Die 1-Norm gibt nicht die exakte Distanz wieder, ist aber ein guter Ersatz, denn es
werden insbesondere Reihenfolgen der Vektoren (also welcher Vektor ist weiter weg als ein anderer)
richtig wiedergegeben.

Diese Idee l"asst sich allerdings erheblich verbessern, indem wir einen Basiswechsel vollziehen in
unseren Projektionsraum plus den restlichen dazugeh"origen orthonormalen Vektoren. Diese restlichen
orthonormalen Vektoren bilden die Freiheitsgrade unserer Tiefenfunktion, also projizieren wir unser
\(\vec{x}\) im Endeffekt auf diese und erhalten damit automatisch die 1-Norm in unserer neuen
Basis. Die Projektionen m"ussen dann nur noch aufsummiert werden.

Unsere Tiefenfunktion \(d_1(\cdot,\cdot)\) lautet also:

	\begin{equation}
		d_1(\vec{x}, p(\vec{x})) = \sum_{k=1}^{j}{\lrangle{ \vec{n}_k, \vec{x} - p(\vec{x}) }},
		\quad j = n -m
	\end{equation}

mit \(\vec{n}\) als restliche zu unserer Projektionsebene \(X\) orthonormale Basisvektoren.

Das formen wir erneut um zu einer Matrix:

	\begin{equation}
		\iff d_1(\vec{x}, p(\vec{x})) = \lrangle{ \vec{x} - p(\vec{x}), \sum_{k=1}^{j}{\vec{n}_k} }
	\end{equation}

	\begin{equation}
		\iff d_1(\vec{x}, p(\vec{x})) = (\sum_{k=1}^{j}{\vec{n}_k})^T (\vec{x} - p(\vec{x}))
	\end{equation}

F"ur die Projektionsfunktion setzen wir die Matrixabbildung ein (mit \(E\) als Einheitsmatrix):

	\begin{equation}
		\iff d_1(\vec{x}, p(\vec{x})) = (\sum_{k=1}^{j}{\vec{n}_k})^T (\vec{x} - P\vec{x})
	\end{equation}

	\begin{equation}
		\iff d_1(\vec{x}, p(\vec{x})) = (\sum_{k=1}^{j}{\vec{n}_k})^T (E - P) \vec{x}
	\end{equation}

Und wir erhalten unsere Tiefenmatrix \(T\), die wir als letzte Zeile an unsere Projektionsmatrix
\(P\) anh"angen k"onnen:

	\begin{equation}
		T = (\sum_{k=1}^{j}{\vec{n}_k})^T (E-P), \quad j = n - m
	\end{equation}

Also m"ussen alle orthonormalen Restvektoren unseres Projektionsraums \(X\) nur aufsummiert und
anschlie"send transponiert werden.

Diese Tiefenfunktion hat zus"atzlich einen sehr praktischen Vorteil: Ist \(m = n - 1\), also
\(j = 1\), so f"allt die Tiefenfunktion mit der Hesse'schen Normalform zusammen und gibt dann sogar
den euklidischen Abstand wieder!
Da nur noch ein orthonormaler Vektor vorhanden ist, f"allt die Summe weg:

	\begin{equation}
		d_1(\vec{x}, p(\vec{x})) = \vec{n}^T \vec{x} = \lrangle{ \vec{n}, \vec{x} - P\vec{x}} =
		d_2(\vec{x}, p(\vec{x}))
	\end{equation}

\end{document}

